% A sample beamer presentation showing 
% what the UNCMath theme looks like.
%
% Key features of the template:
%
% - Little excess colors
% - Using cleaner itemize/enumerate (no "spheres" or "balls")
% - Simple color scheme; shades of "Carolina blue", grayscale
% - small UNC logo in upper right of every slide.
% - removal of some of the ugly default beamer things:
% -- navigation menus
% -- date in footer
% -- page numbering in footer
%
% This is all with the purpose of improving the 
% "usable" space with beamer, and removing 
% the doodads that most people never 
% use or look at when they present or 
% watch someone with a beamer presentation.
%
% First version created by Manuchehr Aminian, April 2017.
% Feel free to modify and share, make suggestions.
% 
% There's no need to credit me, but it would 
% make me happy to know that people used it. 
% Feel free to buy me food or drinks, though.
%

\documentclass{beamer}

\usepackage{graphicx}
%\usepackage{epstopdf}
\usepackage{color}
\usepackage{textcomp}

% Import the UNCMath theme.
% UNC is the default color theme.
\usetheme{UNCMath}

% Uncomment to use a different color scheme.
%\definecolor{darkpurple}{rgb}{0.2,0,0.2}
%\setbeamercolor*{structure}{fg=black,bg=darkpurple}

\logo{
\includegraphics[height=3em]{oldwell_cmyk}
}

% A macro for making the title frame.
% Removes the bottom bar and logo temporarily.
% If you don't want these in other frames, 
% you could try mimicking this.
\newcommand{\titleframe}{
{
\setbeamertemplate{headline}{}
\setbeamertemplate{footline}{}
\begin{frame}
\titlepage
\end{frame}
}
}

%\renewcommand*{\bibfont}{\scriptsize}


%%%%%%%%%%%%%%%%%%%%%%%%%%%%%%%%%%%%%%%%%%
%
% Information for the title frame.
%
\title{DeepStock}
\subtitle{stock price prediction with deep learning}
\author[Siyang Jing \& Jiacheng Tian \& Jiyu Xu \& Yuhui Huang]{
Siyang Jing, Jiacheng Tian, Jiyu Xu, Yuhui Huang
}
\institute{University of North Carolina, Chapel Hill}
\date{Oct 10th, 2018}

%%%%%%%%%%%%%%%%%%%%%%%%%%%%%%%%%%%%%%%%%%
%
% Start of document.
%
\begin{document}

% This is the titleframe, do it this way 
% so that you don't have the logo on the title page.
\titleframe


\begin{frame}
\frametitle{Background and Data}
\centering
\begin{tabular}{|c|c|} \hline
Previous\footnotemark & Our Study \\ \hline
Low Frequency & High Frequency \\
(Daily) & (5-minute) \\ \hline
Simple architecture & Recurrent Neural Network \\ \hline
Price & Price and volume \\ \hline
Simple data processing & Sophisticated feature engineering \\ \hline
Offline learning & Online learning \\ \hline
\end{tabular}
\vspace{1em}
\begin{itemize}
\item Minute-level US equity pricing and volume data since 2000, collected using Google Finance API. Processed with five-minute window.
\item 50 largest stocks as ranked by market capitalization. 
\item Stocks are correlated, so we will use 600 dimensional lagged stock returns (50 stocks and 12 lagged returns) as raw level input data.
\end{itemize}

\footnotetext[1]{According to a survey on the previous researches of stock price prediction \cite{Kearns2013}.}

\end{frame}

\begin{frame}
\frametitle{Model and Method}
Our model follows the research of Chong et al.\cite{Chong2017}:
\begin{align*}
\mathbf{r}_{t} & =[r_1,r_2,...,r_{50}] \\
\mathbf{R}_t & = [\mathbf{r}_t,\mathbf{r}_{t-1},...,\mathbf{r}_{t-12}]^T \\
\mathbf{r}_{t+1} & =f\circ\phi(\mathbf{R}_t)
\end{align*}

\begin{columns}
\begin{column}{0.5\textwidth}
\setlength{\partopsep}{0pt}
$\phi$:
\begin{itemize}
	\item Data \textrightarrow\, Features, e.g. mean, variance, etc.
	\item PCA, autoencoder, RBM, and other unsupervised learning techniques.
\end{itemize}
\end{column}
\begin{column}{0.5\textwidth}
\setlength{\partopsep}{0pt}
$f$:
\begin{itemize}
	\item Predictor function.
	\item Recurrent Neural Network (RNN), state of the art for financial time series data analysis \cite{Abe2018}.
\end{itemize}
\end{column}
\end{columns}

%\footnotetext{We will try to include volume data later.}
\end{frame}

\begin{frame}
\frametitle{Evaluation and Promise}

\begin{itemize}
\item \textbf{Measurements:} normalized mean squared error, mean absolute error, root mean squared error, and mutual information.\\
\item \textbf{Bootstrap analysis:}  will be performed on assessment of the accuracy of the estimator by random resampling with replacement from the original dataset.
\item \textbf{Sensitivity:} of the result with respect to each feature and each stock will also be analyzed.
\end{itemize}

\vspace{2em}
\setlength\parindent{20pt}
We believe an accurate prediction for stock price will lay a solid ground for a successful trading strategy. It will also shed light on the research of economics, finance, behavioral science, and mathematics.

\end{frame}
%
\begin{frame}
\frametitle{Thanks!}

These slides are modified from maminian's template\cite{UNCbeamer}.
\vspace{1em}

%\bibliographystyle{ieeetr}
\bibliography{presentation}

\begin{figure}
\centering
\includegraphics[width=0.5\linewidth]{oldwell_cmyk}
\end{figure}
\end{frame}
%

\end{document}
