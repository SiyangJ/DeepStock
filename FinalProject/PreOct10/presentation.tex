% A sample beamer presentation showing 
% what the UNCMath theme looks like.
%
% Key features of the template:
%
% - Little excess colors
% - Using cleaner itemize/enumerate (no "spheres" or "balls")
% - Simple color scheme; shades of "Carolina blue", grayscale
% - small UNC logo in upper right of every slide.
% - removal of some of the ugly default beamer things:
% -- navigation menus
% -- date in footer
% -- page numbering in footer
%
% This is all with the purpose of improving the 
% "usable" space with beamer, and removing 
% the doodads that most people never 
% use or look at when they present or 
% watch someone with a beamer presentation.
%
% First version created by Manuchehr Aminian, April 2017.
% Feel free to modify and share, make suggestions.
% 
% There's no need to credit me, but it would 
% make me happy to know that people used it. 
% Feel free to buy me food or drinks, though.
%

\documentclass{beamer}

\usepackage{graphicx}
%\usepackage{epstopdf}

% Import the UNCMath theme.
% UNC is the default color theme.
\usetheme{UNCMath}

% Uncomment to use a different color scheme.
%\definecolor{darkpurple}{rgb}{0.2,0,0.2}
%\setbeamercolor*{structure}{fg=black,bg=darkpurple}

\logo{
\includegraphics[height=3em]{oldwell_cmyk}
}

% A macro for making the title frame.
% Removes the bottom bar and logo temporarily.
% If you don't want these in other frames, 
% you could try mimicking this.
\newcommand{\titleframe}{
{
\setbeamertemplate{headline}{}
\setbeamertemplate{footline}{}
\begin{frame}
\titlepage
\end{frame}
}
}


%%%%%%%%%%%%%%%%%%%%%%%%%%%%%%%%%%%%%%%%%%
%
% Information for the title frame.
%
\title{DeepStock}
\subtitle{COMP562 Final Project}
\author[Siyang Jing \& Jiacheng Tian \& Jiyu Xu \& Yuhui Huang]{
Siyang Jing, Jiacheng Tian, Jiyu Xu, Yuhui Huang
}
\institute{University of North Carolina, Chapel Hill}
\date{Oct 10th, 2017}

%%%%%%%%%%%%%%%%%%%%%%%%%%%%%%%%%%%%%%%%%%
%
% Start of document.
%
\begin{document}

% This is the titleframe, do it this way 
% so that you don't have the logo on the title page.
\titleframe


\begin{frame}
\frametitle{Background and Data}
\begin{itemize}
\item Bullets points are now gray.
\item Theme files cleaned up and organized to fit 
the spirit of the Beamer style file structure.
\item Color schemes are now easily swappable by 
setting the \texttt{structure} color in the header 
of the \texttt{.tex} file 
(an example is given in the comments of this 
presentation).
\item Some beamer elements have been integrated in 
to the color scheme.
\item A few minor tweaks were done to spacing.
\item I still don't know how I feel about the ``infoline" 
footer. Pretty ugly, but it adds some essential color 
right now, and I don't know what I would replace it with 
to fit this role.
\end{itemize}
\end{frame}

\begin{frame}
\frametitle{Formulation of Problem}
Here is another equation:
%
\begin{equation}
\int_{-1}^1 \frac{1}{\sqrt{1-x^2}} dx = \pi.
\end{equation}

\begin{enumerate}
\item hmm...
    \begin{enumerate}
    \item item
    \end{enumerate}
\item ok
    \begin{itemize}
    \item Now the bullets are gray.
    \end{itemize}
\end{enumerate}

\end{frame}
%
\begin{frame}
\frametitle{Method}

The color theme has been streamlined and properly linked 
to many of the beamer objects. For example, with a 
theorem:

\begin{theorem}[Intermediate Value Theorem]
Suppose $f(x) : [a,b] \to \mathbb{R}$ is a 
continuous function, and $f(a)f(b)<0$. Then 
there exists a point $c \in (a,b)$ satisfying 
$f(c)=0$.
\end{theorem}

However, there are still a few ``hard-coded" colors which 
I don't think should be tied to the color scheme. 
Primary text color, for instance.

\end{frame}
%
\begin{frame}
\frametitle{Evaluation and Promise}
If you don't include a frame title, there won't be a horizontal 
rule included, but the logo will still show. 
If you don't want the logo in a frame like this 
either, you need to mimick the behavior done with the 
title page macro defined in the source file.
\end{frame}
%
\begin{frame}
\frametitle{References}

These slides are modified from maminian's template\cite{UNCbeamer}.

\bibliographystyle{ieeetr}
\bibliography{presentation}

\begin{figure}
\centering
\includegraphics[width=0.5\linewidth]{oldwell_cmyk}
\end{figure}
\end{frame}
%

\end{document}
